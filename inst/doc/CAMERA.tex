%\VignetteIndexEntry{Molecule Identification with CAMERA}
%\VignetteKeywords{CAMERA}\\
%\VignettePackage{CAMERA}\\

\documentclass[a4paper,12pt]{article}
\usepackage{hyperref}
\usepackage[table]{xcolor}
\setlength{\parindent}{0cm}

\newcommand{\Robject}[1]{{\texttt{#1}}}
\newcommand{\Rfunction}[1]{{\texttt{#1}}}
\newcommand{\Rpackage}[1]{{\textit{#1}}}
\newcommand{\Rclass}[1]{{\textit{#1}}}
\newcommand{\Rmethod}[1]{{\textit{#1}}}
\newcommand{\Rfunarg}[1]{{\textit{#1}}}

\newcommand{\denovo}{{\em de-Novo{}}}
\usepackage{varioref}
\labelformat{figure}{\figurename~#1}
\labelformat{table}{\tablename~#1}


\usepackage{Sweave}
\begin{document}
\title{LC-MS Peak Annotation and Identification with \Rpackage{CAMERA}}
\author{Carsten Kuhl, Ralf Tautenhahn and Steffen Neumann}
\maketitle

\section{Introduction}
%{{{

The R-package \Rpackage{CAMERA} is a {\bf C}ollection of {\bf
A}lgorithms for {\bf ME}tabolite p{\bf R}ofile {\bf A}nnotation. It's
primarily  used for annotation of LC-MS data. Therefore it interacts
directly with processed peak data from \Rpackage{xcms}.

It includes algorithms for annotation of isotope peaks,
adducts and fragments in peak lists generated by \Rpackage{xcms}. Additional
methods cluster mass signals that originate from a single metabolite, based on
rules for mass differences and peak shape comparison \cite{annobird07}.

Based on this annotation results, the molecular composition can be calculated
if the mass spectrometer has a high-enough accuracy for both the mass
and the isotope pattern intensities.

%}}}


\section{Short Background}
%{{{
For soft ionisation methods such as LC/ESI-MS, different adducts (e.g.
$[M+K]^+$, $[M+Na]^+ $) and fragments (e.g.  $[M-C_3H_9N]^+$,
$[M+H-H_20]^+ $) occur. Depending on the molecule having an intrinsic
charge, $[M]^+$ may be observed as well. In most cases a substance generates a
bulk of different ions. There interpretation is time consuming, especially if
substances co-elute. Therefore deconvolution, which separates the different
substances and discovery of the ion species is necessary. 

The working with CAMERA to solves these problems is demonstrated in the next
chapters.
%}}}

\section{Processing with \Rpackage{CAMERA}}

\subsection{Preprocessing with \Rpackage{xcms}}
\label{preprocess}

CAMERA needs as input an \Rclass{xcmsSet} object that is processed with your
favorite parameters. For an example see below:
\begin{verbatim}
  library(CAMERA)
  #Single sample example
  file <- system.file('mzdata/MM14.mzdata', package = "CAMERA")
  xs <- xcmsSet(file,method="centWave",ppm=30,peakwidth=c(5,10))

  #Multiple sample
  library(faahKO)
  filepath <- system.file("cdf", package = "faahKO")
  xsg <- group(faahko)
  xsg <- fillPeaks(xsg)
\end{verbatim}

After the preprocessing we create an CAMERA object, which is called xsAnnotate
or in short xsa. 
\begin{verbatim}
  library(CAMERA)
  xsa <- xsAnnotate(xs)
\end{verbatim}
Depending on your analysis the upcoming workflows may differ at this point and
we start with the description of the annotation workflow. Afterwards we
demonstrate the wrapper functions and how to interpret the results.


\subsection{Annotation Workflow}

The CAMERA annotation procedure can be split into two parts: We want to
answer the questions which peaks occur from the same molecule and
secondly compute its exact mass. Therefore CAMERA annotation workflow contains
four different functions:
\begin{enumerate}
 \item peak grouping after retention time (\Rfunction{groupFWHM})
 \item peak group verification with peakshape correlation
(\Rfunction{groupCorr})
\end{enumerate}
Both methods separate peaks into different groups, which we define as
"pseudospectra". Those pseudospectra can consists from one up to 100 ions,
depending on the molecules amount and ionizability.
Afterwards the exposure of the ion species can be performed with:
\begin{enumerate}
 \item annotation of possible isotopes (\Rfunction{findIsotopes})
 \item annotation of adducts and calculating hypothetical masses for the group
(\Rfunction{findAdducts})
\end{enumerate}
This workflow results in a data-frame similar to a \Rpackage{xcms} peak table,
that can be easily stored in a comma separated table
(Excel-readable).

The next section shows some practical examples.

\subsubsection{Working with single sample}
Let's come to the practical work. Here we have a single sample file either in
positive or negative ionization mode.
The \Rclass{xcmsSet} was created as shown in section \ref{preprocess}.

\begin{verbatim}
  # Create an xsAnnotate object
  an   <- xsAnnotate(xs)
  # Group after RT
  anF  <- groupFWHM(an, perfwhm = 0.6)
  # Annotate isotopes
  anI  <- findIsotopes(anF, mzabs = 0.01)
  # Verify grouping
  anIC <- groupCorr(anI, cor_eic_th = 0.75)
  #Annotate adducts
  anFA <- findAdducts(anIC, polarity="positive")
 \end{verbatim}

In the above example, we create the pseudospectra with the retentiontime
information. The \Rfunarg{perfwhm} parameter defines the window width, which is
used for matching. Lower it for a smaller windows or set it to a  higher value,
if the retention time varies. This step generate 14 pseudospectra.

Afterwards we annotate isotopic peaks, with \Rfunarg{mzabs} as allowed m/z
error. In this example we find 36 isotope peaks, with means the number of
$[M+1],[M+2],\ldots$ ions. This isotope informations are useful in the next step,
where for every peak in one pseudospectra a pairwise EIC correlation is performed. If
the correlation value between two peaks is higher than the threshold
\Rfunarg{cor\_eic\_th} it will stay in the group, otherwise both are separated.
If the peaks are annotated isotope ions, they will not be divided. This seperates
our 14 pseudospectra into 35.
Additionally we could set an constraint that primary adducts e.g. $[M+H]$
and $[M+Na]$ stay together, even if the correlation value is small. To do so
call the function with an polarity parameter like in \Rfunction{findAdducts}
(groupCorr (..., polarity="positive"), which only seperates the 14 into 32
groups. The difference is only small, three single ions were not sorted out,
but as shown in \ref{fig:groupCorr} it is important. In the upper
example an annotation was possible in contrast to the lower one, because the
$[M+Na]$ was put into his own group. So we strongly encourage the use of the
polarity mode.

